\documentclass[sans,mathserif,dvipsnames]{beamer}
\mode<presentation>
\usepackage{pgfpages}
%\setbeameroption{show notes}
%\setbeameroption{show notes on second screen=right}
\setbeamertemplate{footline}[frame number]
\usepackage{amsmath,amssymb}
\usepackage{stmaryrd}
\usepackage{proof}
\usepackage{graphicx}
\usepackage[sans]{dsfont}
\usepackage{rotating}
\usepackage{array}
\usepackage{bbding}
\usepackage{tikz}
\usetikzlibrary{calc,shapes.multipart,chains,arrows,positioning,decorations.markings,shapes,shapes.geometric}


\tikzstyle{terminal} = [rectangle, draw, text width=8em, text centered, rounded corners=1em, minimum height=2em]
\tikzstyle{action} = [draw,rectangle, text width=2cm, minimum width=1cm, text centered, minimum height=2.5em, inner sep=0.1cm, outer sep=0cm]
\tikzstyle{choice} = [draw,diamond,aspect=2,text width=2cm, minimum width=1cm, text centered, minimum height=2.5em, inner sep=0.1cm,outer sep=0cm]

\newcommand{\theauthor}[2]{%
\includegraphics[width=0.6\textwidth]{#2}\\#1
}%

\newcommand{\thepicture}[1]{%
\includegraphics[width=0.8\textwidth]{#1}
}%

\begin{document}

\title{Caper: Automatic Verification with Concurrent Abstract Predicates}
\author{Thomas Dinsdale-Young \and \textbf{Pedro da Rocha Pinto} \and Kristoffer Just Andersen \and Lars Birkedal}
\date{}

\begin{frame}
\maketitle
\end{frame}

\begin{frame}{Concurrent Abstract Predicates}
  \begin{itemize}
    \item CAP provides a way of specifying arbitrary knowledge and rights as resources
    \item A region provides an abstraction over some shared resources
    \item The region is equipped with a protocol that determines how threads can update it
    \item Resources called guards determine the roles threads can play in the protocol
  \end{itemize}
\end{frame}

\begin{frame}{Example: Spin Lock}
  \begin{itemize}
    \item A spin lock region has two states: locked (1) and unlocked (0)
    \item Any thread can update it from unlocked to locked
    \item Only the thread that acquired the lock can update it from locked to unlocked
  \end{itemize}
\end{frame}

\begin{frame}{Demo: Spin Lock}
\end{frame}

\begin{frame}{What else}
  \begin{itemize}
    \item Can specify multiple guards for a region type
    \begin{itemize}
      \item Singleton
      \item Permissions
      \item Set
      \item Product
      \item Sum
    \end{itemize}
    \item We have applied it to several examples:
    \begin{itemize}
      \item Spin counter, blocking counter, bounded counter
      \item Ticket lock, read writers lock
      \item Fork/Join
      \item Stack
    \end{itemize}
  \end{itemize}
\end{frame}

\begin{frame}{Summary}
 \begin{itemize}
    \item Caper supports reasoning about concurrency through shared regions
    \item Sharing is mediated by protocols, with roles determined by ownership of guard resources
    \item Given the protocol and currently owned guards, Caper computes rely and guarantee relations as part of its symbolic execution
  \end{itemize}
\end{frame}

\begin{frame}{Future work}
  \begin{itemize}
    \item Support for richer abstractions
      \begin{itemize}
        \item Sets, lists, etc.
        \item Necessary for many interesting data structures
      \end{itemize}
    \item Improved proof/proof-search output
      \begin{itemize}
        \item Interactive navigation
        \item Certificate generation for external proof checker
      \end{itemize}
    \item Support for abstract atomicity
  \end{itemize}
\end{frame}

\begin{frame}
``Man's life's a vapor, and full of woes; he cuts a caper, and down he goes'' - Proverb
\end{frame}

\end{document}
