\documentclass[a4paper]{article}

\usepackage{amsmath}
\usepackage{amsthm}
\usepackage{amssymb}
\usepackage{listings}
\usepackage{color}

% Programming language

\newcommand{\pfunction}[4]{\mathtt{function} \ #1(#2) \ \{ \mathtt{var} \ #3 ; \ #4 \}}
\newcommand{\psequence}[2]{#1 ; #2}
\newcommand{\pifelse}[3]{\mathtt{if} \ (#1) \ #2 \ \mathtt{else} \ #3}
\newcommand{\pwhile}[2]{\mathtt{while} \ (#1) \ #2}
\newcommand{\pdowhile}[2]{\mathtt{do} \ #1 \ \mathtt{while} \ (#2)}
\newcommand{\plassign}[2]{#1 := #2}
\newcommand{\passign}[2]{[#1] := #2}
\newcommand{\pderef}[2]{#1 := [#2]}
\newcommand{\pfuncall}[3]{#1 := #2(#3)}
\newcommand{\preturn}[1]{\mathtt{return} \ #1}
\newcommand{\pskip}[0]{\mathtt{skip}}
\newcommand{\pfork}[3]{#1 := \mathtt{fork} \ #2(#3)} 
\newcommand{\pjoin}[2]{#1 := \mathtt{join} \ #2}

\newtheorem*{extension}{Extension}


\lstdefinelanguage{caper}
  {morekeywords={region,guards,interpretation,actions,function,requires,ensures,var,alloc,return,CAS,do,while,if,else},
  sensitive=true,
  morecomment=[l]{//},
  morecomment=[s]{/*}{*/}
  }
\lstset{
  basicstyle=\small,
  keywordstyle=\color{red}
}

\begin{document}
\title{\textsc{Caper}: Design Document}
\author{}

\maketitle

This document specifies \textsc{Caper}, a prototype tool for (semi-)automatic program verification using Concurrent Abstract Predicates (CAP).
Programs will be annotated with pre- and postconditions on functions, loop invariants and other assertions, definitions or proof hints.
\textsc{Caper} will attempt to verify programs against these specifications.

\section{Programming Language}

The \textsc{Caper} programming language has:
\begin{itemize}
  \item functions (procedures) as the basic top-level elements;
  \item function-scoped mutable variables;
  \item a shared heap, mutable with primitive atomic operations;
  \item integer arithmetic;
  \item conditional and loop constructs;
  \item fork/join concurrency.
\end{itemize}

The language 

\section{Program Logic}

The \textsc{Caper} program logic is based on CAP and inspired by iCap and TaDA.
As in CAP, assertions describe resources owned by the local thread and resources belonging to shared regions.
Shared regions are subject to protocols, which determine how their resources may be used.

As in iCap, but unlike in CAP, the state of shared regions are abstracted.
That is, instead of directly specifying the resources currently belonging to the region, an assertion specifies the possible abstract states of a region, which are associated with concrete resources by an interpretation function.
Abstract states of regions are integers.

\begin{extension}
  Permit other datatypes as the abstract states of regions.
\end{extension}

Each region is associated with a protocol that determines how its (abstract) state may change.
In order for a thread to perform a state change, it must have access to a resource (called a guard) associated with the state change in the protocol.
In CAP, the guard is always a non-zero fractional permission to a single named (and parametrised) guard.
In TaDA, guards are elements of an arbitrary separation algebra associated with the region.
\textsc{Caper} takes a middle ground: guards are elements of a separation algebra of a certain form.

\subsection{Guards and Guard Types}

A guard type specifies the separation algebra of guards that is associated with a region.
Guard types are composed from primitive guard types.
Each primitive guard type is associated with a name, which must be unique within the overall guard type.
There are three kinds of primitive guard types:
\begin{itemize}
  \item Unique guards consist of a name and nothing else: there can be only one such guard for a region.
  \item Permission guards consist of a name and a permission value: there may be multiple permission guards for a region, whose permissions compose to give full permission.
  \item Parametrised guards consist of a name together with an integer value: there is one of these guards for each integer.
    The syntax for guards makes it possible to represent the combination of all parametrised guards of a given name, excluding a finite set.
\end{itemize}

\begin{extension}
  Permit combinations of parameters and permissions.
\end{extension}

There are two operators for combining guard types: product and sum.
The product of two guard types $a \times b$ can be thought of as the set of pairs of guards from $a$ and guards from $b$.
(This corresponds to a direct product on the separation algebras.)

The sum of two guard types $a + b$ contains all guards from $a$ and all guards from $b$.
(This can be thought of as the (disjoint/discriminated) union of guard types $a$ and $b$.)
However, we identify the ``full'' guards of $a$ and $b$.

The sum construction allows us to specify rich forms of capabilities, such as deny-guarantee or insert-remove-definite.

\subsection{Protocols}

Here are some examples of protocol specifications.

For a lock:
\[
\begin{array}{r@{\ :\ } c @{\ \rightsquigarrow\ } c}
  A [ \_ ] & 0 & 1 \\
  B & 1 & 0
\end{array}
\]
$A [ \_ ]$ indicates that we have some (unspecified but non-zero) permission on $A$.

\[
\begin{array}{c @{\ |\ } r@{\ :\ } c @{\ \rightsquigarrow\ } c}
  n > 0 & A (n) & n & -n \\
  n > 0 & B & -n & n+1 \\
  n > 0 & B * C & 0 & n \\
\end{array}
\]

Logical variables may occur in protocol specifications.
We can stipulate constraints on the logical variables as side-conditions.

\section{First Example}

We shall use a simple compare-and-swap lock as the motivating example for the \textsc{Caper} tool.
The code, together with annotations, is shown in Figure~\ref{fig:caslock}.

\begin{figure}
\begin{center}
\begin{lstlisting}[language=caper,mathescape,columns=flexible]
// Compare-and-swap lock

region Lock(r,x) {
  guards %LOCK * UNLOCK;
  interpretation {
    0 : x $\mapsto$ 0 * r@UNLOCK;
    1 : x $\mapsto$ 1;
  }
  actions {
    LOCK[_] : 0 $\rightsquigarrow$ 1;
    UNLOCK : 1 $\rightsquigarrow$ 0;
  }
}

function makeLock()
  requires emp;
  ensures Lock(r,ret,0) * r@LOCK[1]; {
    var v;
    v := alloc(1);
    [v] := 0;
    return v;
}

function lock(x)
  requires Lock(r,x,_) * r@LOCK[p];
  ensures Lock(r,x,1) * r@(LOCK[p] * UNLOCK); {
    var b;
    do {
        b := CAS(x, 0, 1);
    } while (b = 0);
}

function unlock(x)
  requires Lock(r,x,1) * r@UNLOCK;
  ensures Lock(r,x,_); {
    [x] := 0;
}
\end{lstlisting}
\end{center}
\caption{Compare-and-swap lock example.}
\label{fig:caslock}
\end{figure}

It may be necessary to add an invariant assertion to the loop.

\section{Grammar}

\[
\begin{array}{rcl}
  F & ::= & \pfunction{\mathtt{f}}{\mathtt{x}}{\mathtt{x}}{C} \\
  C & ::= & \psequence{C}{C} \ | \ \pifelse{B}{C}{C} \ | \ \pwhile{B}{C} \ | \ \pdowhile{C}{B} \ | \ \\
  & & \plassign{\mathtt{x}}{E} \ | \ \pderef{\mathtt{x}}{E} \ | \ \passign{E}{E} \ | \ \pfuncall{\mathtt{x}}{\mathtt{f}}{E} \ | \ \preturn{E} \ | \ \pskip \ | \ \\
  & & \pfork{\mathtt{x}}{\mathtt{f}}{E} \ | \ \pjoin{\mathtt{x}}{E} \\
  E & ::= & \mathtt{x} \ | \ n \ | \ ( E ) \ | \ E + E \ | \ E - E \ | \ E * E \ | \ E / E\\
  B & ::= & \mathtt{true} \ | \ \mathtt{false} \ | \ ( B ) \ | \ E = E \ | \ E \ \text{!=} \ E \ | \ E > E \ | \ \\
  & & E \ge E \ | \ E < E \ | \ E \le E \ | \ E \ \mathtt{or} \ B \ | \ B \ \mathtt{and} \ B \ | \ \mathtt{not} \ B
\end{array}
\]

\end{document}
